This section should provide a critical examination of the constraints and potential weaknesses inherent in your study. This includes, but is not limited to, methodological limitations (e.g., sample size, data collection methods, tools used), external validity issues (e.g., generalizability of the findings to other contexts or populations), and any biases that may have influenced the results. It is important to discuss how these limitations might have affected the outcomes and the interpretation of your findings. Rather than viewing limitations solely as flaws, consider them as areas for future research and opportunities for improving the study design. Being transparent about the limitations will enhance the credibility of your work and help readers understand the breadth and depth of your research contributions. When detailing these limitations, ensure to provide specific examples and, where possible, suggest ways in which future studies could address or mitigate these issues.