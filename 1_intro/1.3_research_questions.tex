After identifying the gap in the literature, this section should clearly articulate how you intend to address this gap by formulating one to three specific research questions. Begin by describing the overarching objective of your study. This objective should encapsulate the primary goal of your research and align with the gap you have identified. Once the objective is outlined, develop concise and focused research questions that reflect the core issues you aim to explore. These questions should be clear, measurable, and achievable within the scope of your thesis.

Each research question should ideally be accompanied by one to three hypotheses that can be tested, preferably using inferential statistics. Hypotheses serve as tentative answers to your research questions, providing a basis for testing and validation. These hypotheses should be specific, testable statements that predict an expected relationship between variables. By framing your research questions and hypotheses clearly, you set the stage for a structured and methodical investigation.

For example:
\begin{enumerate}
    \item \textbf{Research Question:} How does [specific variable] influence [specific outcome] in [specific context]?
    \begin{itemize}
        \item \textbf{Hypothesis 1:} [Specific variable] will be positively correlated with [specific outcome].
        \item \textbf{Hypothesis 2:} [Specific intervention] will moderate the relationship between [specific variable] and [specific outcome].
    \end{itemize}
    \item \textbf{Research Question:} What are the effects of [specific intervention] on [specific variable]?
    \begin{itemize}
        \item \textbf{Hypothesis 1:} [Specific intervention] will significantly increase [specific variable].
        \item \textbf{Hypothesis 2:} The effect of [specific intervention] on [specific variable] will vary based on [specific moderating variable].
    \end{itemize}
    \item \textbf{Research Question:} How do [specific group] perceive [specific phenomenon]?
    \begin{itemize}
        \item \textbf{Hypothesis 1:} [Specific group] will have a more positive perception of [specific phenomenon] compared to [another group].
        \item \textbf{Hypothesis 2:} Perception of [specific phenomenon] will be influenced by [specific factor].
    \end{itemize}
\end{enumerate}

By carefully crafting the "Objective and Research Questions" section and pairing each research question with clearly defined hypotheses, you set a clear and focused foundation for your study. This approach ensures that each subsequent chapter contributes towards testing these hypotheses and answering your research questions, thereby enhancing the rigor and coherence of your thesis.
