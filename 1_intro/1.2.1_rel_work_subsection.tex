You can structure the literature review in connected subsections in order to present different research streams of the literature. The literature review should contain the main literature until the recent year. You can divide the different research streams in sub.parts and start with the oldest literature (10 years ago, if later, it should be very relevant) and should also include literature up to date. 

Imagine the literature review like a \textbf{funnel} where you start more broad and brake the research streams down to your exact research topic. 

For the literature review, research literature reviews which are published in the research domain first, there you already get an impression what is the most relevant literature and the different research streams. That does not necessarily mean you need to read 100 papers, but to have an idea who contributed what, therefore, previous literature reviews can really be helpful in saving some time. Most of the time you cannot find literature reviews for your exact research question. Therefore, look also for related topics. 

Best would be if you make a table by structuring the literature for yourself while doing the literature research and write the chapter asap. Later on after the implementation phase you will have forgotten most of the literature. By writing this during the literature research you can definitely save some time!

You can also add the overview table to the literature review chapter, if your topic is complex and the table has some added value to the comprehensibility of the topic. An example table is added in the \textit{long\_table\_example.tex} file.
