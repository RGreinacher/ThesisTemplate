All chapters should have a short introduction what the chapter is about and how it is structured, such that for the reader is always clear what to come. Please structure your thesis such, that the argumentation follows a red line an the sequencing makes logically sense! If the thesis has a clear structure and follows an understandable logic, has no grammar issues or formatting errors, then this gives already a positive impression! 

The introduction is structured in three different parts:

\begin{enumerate}
    \item Motivation for the topic of the thesis and should be framed such, that the reader is promptly interested in the topic. The reader wants to understand why they're reading your work in the first place and what to expect.
    \item The second part is about the already existing works in this field. Adjacent topics can also be mentioned. This related work part outlines the field, how things are done here, what results are already known. This part ends with a detailed paragraph about what hasn't been done yet, what is still missing, what's the \textit{gap in the literature} or what's \textit{missing in the body of evidence}. From this, you can directly transition into the last part:
    \item Here, you describe how you want to address this gap in the literature, what you work contributes to the existing knowledge; what you plan to do. Here goes your research questions.
\end{enumerate}

All these things together might make up 20\% to 25\% of your thesis.
Sometimes, when the structure of the thesis is complicated you can add a graph to visualize the structure to make it more comprehensible. 

